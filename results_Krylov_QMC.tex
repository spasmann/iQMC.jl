% !TEX root = Krylov_QMC.tex
\section{Computational Results}
\label{sec:results}
In this section we consider an example from
\cite{cesinh}. The formulation of the transport
problem is taken from \cite{ctk:jeff1}. The equation for the angular
flux \(\psi\) is

\begeq
\label{eq:transportgs}
\mu \frac{\partial \psi}{\partial x} (x,\mu) + \Sigma_t(x) \psi(x,\mu) =
\frac{1}{2} \left[ \Sigma_s(x) \int_{-1}^1 \psi(x, \mu') \dmup + q(x) \right]
 \mbox{ for } 0 \le x \le \tau
\endeq

The boundary conditions are

\[
\psi(0, \mu) = \psi_l(\mu), \mu > 0; \psi(\tau, \mu) = \psi_r(\mu),
\mu < 0.
\]

\subsection{Multigroup Equations}
In general geometry the multigroup equations are 
\begin{equation}\label{eq:MG}
\mu  \frac{\partial \psi_g}{\partial x} (x,\mu) + \Sigma_{t,g}(x) \psi_g(x,\mu) =
\frac{1}{2} \sum_{g'=1}^G \Sigma_{s,g'\rightarrow g}(x) \int_{-1}^1 \psi_{g'}(x, \mu') \dmup + \frac{q_g(x)}{2} \quad g=1,\dots,G.
\end{equation}The boundary conditions are
\[
\psi_g(0, \mu) = \psi_{l,g}(\mu), \mu > 0; \psi_g(\tau, \mu) = \psi_{r,g}(\mu),
\mu < 0.
\]
In matrix form, these equations are
\begin{equation}\label{eq:MGmat}
\mu  \frac{\partial \vec{\psi}}{\partial x} (x,\mu) + \underline{\Sigma}_{t}(x) \vec{\psi}(x,\mu) =
\frac{1}{2}  \underline{\Sigma}_{s}(x) \int_{-1}^1 \vec{\psi}(x, \mu') \dmup + \frac{\vec{q}(x)}{2},
\end{equation}
where
\begin{equation}\label{eq:vecs}
\vec{\psi} = (\psi_1, \psi_2, \dots, \psi_G)^\mathrm{T}, \qquad \vec{q} = (q_1, q_2, \dots, q_G)^\mathrm{T}, 
\end{equation}
\begin{equation}\label{eq:MatricesT}
 \underline{\Sigma}_{t}(x)  = \begin{pmatrix} \Sigma_{t,1}(x) & 0 & \dots\\
 0 & \Sigma_{t,2}(x) & 0 \dots \\
 \vdots & & \ddots\\ 
 0 & \dots & 0 & \Sigma_{t,G}(x) 
 \end{pmatrix}, 
\end{equation}
and
\begin{equation}\label{eq:MatricesS}
 \underline{\Sigma}_{s}(x)  = \begin{pmatrix} \Sigma_{s,1\rightarrow 1}(x) & \Sigma_{s,2\rightarrow 1}(x)  & \dots & \Sigma_{s,G\rightarrow 1}(x) \\
 \Sigma_{s,2\rightarrow 1}(x) & \Sigma_{s,2\rightarrow 1}(x)  & \dots & \Sigma_{s,G\rightarrow 2}(x) \\
 \vdots & \vdots & & \vdots\\
 \Sigma_{s,G\rightarrow 1}(x) & \Sigma_{s,G\rightarrow 1}(x)  & \dots & \Sigma_{s,G\rightarrow G}(x) \\
 \end{pmatrix},.
\end{equation}

\subsection{Source Iteration and Linear Solvers}
\label{subsec:source}

Source iteration is Picard iteration for the fixed point problem
\[
\phi = \cals(\phi, q, \psi_l, \psi_r)
\]
To use other solvers we must convert to a linear system via
\[
\calk(\phi) = \cals(\phi, 0, 0, 0) \mbox{ and }
f = \cals(0, q, \psi_l, \psi_r)
\]
to get
\[
A \phi \equiv (I - \calk) \phi = f,
\]
which we can send to a linear solver.

In the computations we use the problem from \cite{cesinh}
\[
\tau=5, \Sigma_s(x) =\omega_0 e^{-x/s},  \Sigma_t(x) = 1, q(x) = 0, \psi_l(\mu) = 1, \psi_r(\mu) = 0,
\]
and consider two cases $s=1$ and $s=\infty$ 

\clearpage

\subsection{QMC and Krylov Linear Solvers}
\label{subsec:krylov}


The linear and nonlinear solvers come from the Julia package
\href{https://github.com/ctkelley/SIAMFANLEquations.jl}{SIAMFANLEQ.jl}
\cite{ctk:siamfanl}. The documentation for these codes is in the
\href{https://github.com/ctkelley/NotebookSIAMFANL}{Juila notebooks}
\cite{ctk:notebooknl} and the book \cite{ctk:fajulia}
that accompany the package. 


We solve the QMC linear problem with N=2048 and Nx= 100. 
We use two krylov methods \cite{ctk:roots}, GMRES \cite{gmres} and
Bi-CGSTAB \cite{bicgstab}.  Figures~\ref{fig:easy} and 
\ref{fig:hard} show that
the Krylov iterations take fewer than half of the number
of transport sweeps that Picard iteration required.

\vspace*{.25in}

\begin{figure}
\centerline{
\includegraphics[width=3.5in]{FIGURES/seqone.pdf}
}
\caption{\label{fig:easy} $s=1$}
\end{figure}

\begin{figure}
\centerline{
\includegraphics[width=3.5in]{FIGURES/seqinf.pdf}
}
\caption{\label{fig:easypdf} $s=1$}
\end{figure}

\begin{figure}[h]
  \centering
  \includegraphics[trim = 10mm 0mm 15mm 15mm, width=70mm]{FIGURES/seqinf.png}
  \caption{$s = \infty$}
  \label{fig:hard}
\end{figure}

\clearpage

\subsection{Validation and calibration study}
\label{validation-and-calibration-study}

We conclude this section with a validation study. We compare the
QMC results with the results from \cite{cesinh}. The results
in \cite{cesinh} are exit distributions and are accurate to 
six figures. We have duplicated those results with an $Sn$ computation
on a fine angular and spatial mesh.

{\bf Sam, Ryan, should we use more or different values of $N$ and $Nx$?}

For $N = 1000$ and $Nx=100$ we obtain the cell-average fluxes from
the QMC approximation. We then use a single Sn transport sweep to recover
the exit distributions from the QMC cell-average fluxes. We report
the results and the corresponding results from \cite{cesinh} in 
Tables~\ref{tab:cesone} and \ref{tab:cesinf}.

The exit distributions, as is clear from Table~\ref{tab:cesone}
can vary by five orders of magnitude. Even so, the results from QMC
agree with the benchmarks to roughly two figures.

\begin{table}[h]
\centering
\caption{Exit Distributions: $s = 1$}
\label{tab:cesone}
\centerline{
\begin{tabular}{lllll}
 & \multicolumn{2}{c}{Garcia/Siewert}
 & \multicolumn{2}{c}{QMC}\\
\hline
\hline
$\mu$ &$\psi(0, -\mu)$ &$\psi(\tau, \mu)$ &$\psi(0, -\mu)$ &$\psi(\tau, \mu)$ \\
\hline
5.00e-02 &  5.89664e-01 &  6.07488e-06 &  5.71197e-01 &  5.85487e-06   \\
1.00e-01 &  5.31120e-01 &  6.92516e-06 &  5.22137e-01 &  6.66741e-06   \\
2.00e-01 &  4.43280e-01 &  9.64232e-06 &  4.41567e-01 &  9.25261e-06   \\
3.00e-01 &  3.80306e-01 &  1.62339e-05 &  3.81029e-01 &  1.54416e-05   \\
4.00e-01 &  3.32964e-01 &  4.38580e-05 &  3.34673e-01 &  4.09691e-05   \\
5.00e-01 &  2.96090e-01 &  1.69372e-04 &  2.98224e-01 &  1.57373e-04   \\
6.00e-01 &  2.66563e-01 &  5.73465e-04 &  2.68871e-01 &  5.35989e-04   \\
7.00e-01 &  2.42390e-01 &  1.51282e-03 &  2.44749e-01 &  1.42448e-03   \\
8.00e-01 &  2.22235e-01 &  3.24369e-03 &  2.24583e-01 &  3.07431e-03   \\
9.00e-01 &  2.05174e-01 &  5.96036e-03 &  2.07478e-01 &  5.67991e-03   \\
1.00e+00 &  1.90546e-01 &  9.77123e-03 &  1.92789e-01 &  9.35351e-03   \\
\hline
\end{tabular}
}
\end{table}


\begin{table}[h]
\centering
\caption{Exit Distributions: $s = \infty$}
\label{tab:cesinf}
\begin{tabular}{lllll}
 & \multicolumn{2}{c}{Garcia/Siewert}
 & \multicolumn{2}{c}{QMC}\\
\hline
$\mu$ &$\psi(0, -\mu)$ &$\psi(\tau, \mu)$ &$\psi(0, -\mu)$ &$\psi(\tau, \mu)$ \\
\hline
5.00e-02 &  8.97798e-01 &  1.02202e-01 &  8.47454e-01 &  1.00663e-01   \\
1.00e-01 &  8.87836e-01 &  1.12164e-01 &  8.52822e-01 &  1.10325e-01   \\
2.00e-01 &  8.69581e-01 &  1.30419e-01 &  8.47710e-01 &  1.29064e-01   \\
3.00e-01 &  8.52299e-01 &  1.47701e-01 &  8.35879e-01 &  1.46849e-01   \\
4.00e-01 &  8.35503e-01 &  1.64497e-01 &  8.22291e-01 &  1.64034e-01   \\
5.00e-01 &  8.18996e-01 &  1.81004e-01 &  8.08044e-01 &  1.80827e-01   \\
6.00e-01 &  8.02676e-01 &  1.97324e-01 &  7.93459e-01 &  1.97336e-01   \\
7.00e-01 &  7.86493e-01 &  2.13507e-01 &  7.78672e-01 &  2.13625e-01   \\
8.00e-01 &  7.70429e-01 &  2.29571e-01 &  7.63768e-01 &  2.29725e-01   \\
9.00e-01 &  7.54496e-01 &  2.45504e-01 &  7.48818e-01 &  2.45642e-01   \\
1.00e+00 &  7.38721e-01 &  2.61279e-01 &  7.33889e-01 &  2.61361e-01   \\
\hline
\end{tabular}
\end{table}

In Tables \ref{tab:bigtab1} and \ref{tab:bigtabinf} we look at the
relative errors in the QMC exit distributions as compared to a highly
accurate SN result. We compensate for the widely varying scales by tabulating,
for each value of $N$ and $Nx$
\[
R = \max(R^0, R^\tau)
\]
where
\[
R^0 = \max_\mu
\frac{ | \psi^{SN}(0,-\mu) - \psi^{QMC}(0,-\mu) | }{\psi^{SN}(0,-\mu) }
\]
and
\[
R^\tau = \max_\mu
\frac{ | \psi^{SN}(\tau,\mu) - \psi^{QMC}(\tau,\mu) | }{\psi^{SN}(\tau,\mu) }.
\]

{\bf Ryan, for large Nx I see convergence as N increases. Is it clearly
$1/N$? Am I missing something? Am I tabulating the wrong thing?}

\begin{table}[h]
\centering
\caption{Exit Distributions Errors: $s = 1.0$}
\label{tab:bigtab1}
\begin{tabular}{l|lllll} 
Nx \textbackslash N &     1000 &     2000 &     4000 &     8000 &    16000 \\ 
\hline 
50 & 1.41162e-01& 1.36428e-01& 1.34747e-01& 1.35736e-01& 1.35577e-01   \\ 
100 & 7.08438e-02& 6.60744e-02& 6.52017e-02& 6.51605e-02& 6.49914e-02   \\ 
200 & 4.17171e-02& 3.30480e-02& 3.23088e-02& 3.21432e-02& 3.17467e-02   \\ 
400 & 4.55590e-02& 1.73115e-02& 1.63072e-02& 1.61542e-02& 1.58469e-02   \\ 
800 & 4.83754e-02& 1.93087e-02& 1.29178e-02& 8.30117e-03& 7.96562e-03   \\ 
1600 & 5.07584e-02& 2.03691e-02& 1.44681e-02& 4.52388e-03& 4.11350e-03   \\ 
3200 & 5.09694e-02& 2.13418e-02& 1.48086e-02& 2.88667e-03& 2.18194e-03   \\ 
\hline 
\end{tabular} 
\end{table}

\begin{table}[h]
\centering
\caption{Exit Distributions Errors: $s = \infty$}
\label{tab:bigtabinf}
\begin{tabular}{l|lllll} 
Nx \textbackslash N &     1000 &     2000 &     4000 &     8000 &    16000 \\ 
\hline 
50 & 5.95648e-02& 2.42755e-02& 1.36521e-02& 1.29509e-02& 1.22769e-02   \\ 
100 & 5.60749e-02& 2.31030e-02& 1.31680e-02& 6.45949e-03& 6.59550e-03   \\ 
200 & 5.62864e-02& 2.32524e-02& 1.42149e-02& 4.77319e-03& 3.55246e-03   \\ 
400 & 5.30954e-02& 2.17854e-02& 1.48225e-02& 4.73260e-03& 2.05558e-03   \\ 
800 & 7.66264e-02& 1.88155e-02& 1.60082e-02& 4.34610e-03& 1.41402e-03   \\ 
1600 & 5.99376e-02& 2.15675e-02& 1.56784e-02& 4.21636e-03& 1.29138e-03   \\ 
3200 & 5.74319e-02& 1.89482e-02& 2.00195e-02& 3.26688e-03& 1.49649e-03   \\ 
\hline 
\end{tabular} 
\end{table}
