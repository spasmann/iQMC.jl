\section{Implementation}
\label{sec:implementation}


 %%%%%%%%%%%%%%%%%%%%%%%%%%%%%%%%%%%%%%%%%%%%%%%%%%%%%%%%%%%%%%%%%%
 %%%%%%%%%%%%%%%%%%%%%%%%%%%%%%%%%%%%%%%%%%%%%%%%%%%%%%%%%%%%%%%%%%
The program was written in Julia, a scientific computing language that combines the compiler capabilities of C++ and the syntax of Matlab and Python. The code and primary documentation are available here \href{https://github.com/ctkelley/Krylov_QMC} {Krylov\textunderscore QMC} \cite{ctk:krylovqmc}. The linear and nonlinear solvers come from the Julia package \href{https://github.com/ctkelley/SIAMFANLEquations.jl}{SIAMFANLEQ.jl} \cite{ctk:siamfanl}. The documentation for these codes is in the \href{https://github.com/ctkelley NotebookSIAMFANL}{Juila notebooks} \cite{ctk:notebooknl} and the book \cite{ctk:fajulia} that accompany the package. 